\documentclass[main.tex]{subfiles}
\begin{document}

Maybe, life didn't start on Earth. This might sound wild, well, welcome to the Panspermia Hypothesis. This is the idea that life came to Earth by some external agent (usually an asteroid). One “popular” idea is that life originated on Mars (since the environment was drastically different back then) and was deposited to Earth by asteroids. We have rocks from Mars on Earth, so the exchange of materials between the planets is known, it's whether these rocks contained any life forms that needs to be answered. Obviously, a goat wouldn't fare well on a 150 million mile mission through the vacuum of space to visit Earth. However, there are life forms, such as the Tardigrades, that could survive the perilous journey. As much fun as this theory is, what could this mean for Earth. Well, if life truly did originate from Mars, we can potentially look further back in the evolution of life, maybe even find the point when chemistry became biology. This could lead to ground-breaking discoveries in how we view life. If we learn all this new “stuff” about life, we might be able to artificially create forms of life for our own benefit (queue the simulation theory), such as the structures found in bacteria that now make up CRISPR technology. Maybe these new forms of life could lead to unique engineering feats to cure diseases or otherwise enhance our living. Needless to say, this isn't a great selling point since it all hinges on the idea that life began on Mars. However, even if we don't find any convincing evidence that life started on Mars, we can still find traces of life on Mars. Sadly, Mars rovers aren't experts in excavation so a human crew will likely be dispatched to dig up some soil and try to make reasonable progress towards finding evidence of life on Mars. This discovery could show the fundamental differences in the development of life on Earth and Mars (provided the Panspermia Hypothesis is wrong). Again, we could exploit this. Some organisms are tough to fight since they've had millennia to develop fighting strategies, but what happens when a new, unknown opponent enters the ring, "Years of academy training wasted!" This could prove very dangerous, however. Introducing a foreign species to fight a formidable foe could backfire by killing off the whole species - a species that may have had an important role in our lives. This area should obviously be strongly regulated to prevent such mishaps, but having access to such an organism could be used safely in a lab to fight off certain bacteria or viruses. At this point, it is just speculation, but (as with the Innovation section), if life is discovered on Mars, I think it is quite probable that this could lead to some practical advancements in biology. 

\end{document}