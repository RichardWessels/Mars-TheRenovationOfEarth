\documentclass[main.tex]{subfiles}
\begin{document}

Now that I have the attention of the US government, let's look at the potential import of resources from Mars. First off, space travel is expensive. Transporting heavy rocks from Mars would not be cheap, so unless you're bringing back holy stones that promise immunity to all in its presence, it won't be worth the trouble. However, hitching a few stones on the back of a return flight for humans is likely practical. Due to the limitations of return cargo, these stones will likely be used for research rather than exploitation. Does that mean that there is nothing to import from Mars? There is one major resource that could be imported, or more realistically be used heavily on Mars - deuterium, an isotope of hydrogen that is commonly used for fusion reactors. Since fusion is the future (for the past 60 years), Mars could act as a testing ground for nuclear reactors. Mars has approximately eight times as much deuterium as Earth\textsuperscript{2}. Since deuterium isn't as heavy as iron, it could be brought to Earth in useful loads that could power the development, and hopeful adoption, of fusion reactors. On the topic of innovation, the advent of useful fusion reactors may be hastened by the push to go to Mars; this added with the “abundance” of deuterium could help make fusion reactors commonplace. For the most part, I don't think mining materials on Mars and sending them to Earth will be a common practice. That doesn't mean that Mars is useless in the resource department. As said, deuterium could be imported at a high cost, and there will likely be some Martian rocks to sell on Earth.

\end{document}