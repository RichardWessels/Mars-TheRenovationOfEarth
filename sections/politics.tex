\documentclass[main.tex]{subfiles}
\begin{document}

Since deuterium mining is a very controversial and heated speaking point, let's move onto the more light-weighted topic of politics. There are two main models when it comes to how a mission to Mars will look in terms of politics: the Sagan model and the Kennedy model. The Sagan model pushes for a worldwide collaborative effort. Different nations collaborate and therefore are brought together. An example of this is the International Space Station (hence the 'International'). Such an effort is favorable as resources can be pooled and it can provide a way to ease tensions between countries (such as the United States and China at the moment). One small issue is that a mission to Mars involves some pretty powerful rockets that need to be developed, and if worldwide collaboration is done, then information about the rocket's manufacturing process wouldn't be too hard to come by and therefore can be used by the wrong people. This is a minor point because only a portion of the mission consists of rocket technology, the rest involves habitation, life support, etc. A global collaborative effort may also be far slower than its competitive counterpart, the Kennedy model. This model relies on a national effort that is fueled by patriotism. This model was very useful back in the Moon Race because of the Cold War. Many patriotic citizens greatly supported the space missions because they considered it a point of pride for their democratic and capitalist country. This gave immense motivation that moved space travel advancements by leaps and bounds. Although great for our desire to get to the moon, I don't think the Moon Race helped relations between America and the USSR. If this model is used today with America and China, tensions may dip for the worse. The Kennedy model may not be the best model for the 21st century because it pushes us away from globalism. Since I mentioned the Cold War, let's move to ideologies. This is a very important, yet sensitive, topic. Will the colony of Mars be strongly democratic fueled by capitalism, or will it be another try at communism (or maybe even anarchism). Such extremes aside, I think the Mars colony will begin as a democratic base, with capitalism at its core. With the ongoing innovations in the field of automation, we could see the work done by the settlers as less useful, therefore, socialist measures will seem likely. Depending on the advancements of automation, we could even see a communist Mars (I mean, it is the red planet). This is very radical currently, but if the work done by humans on Mars is insignificant compared to technology, they may not even have to work. They can go play in the decreased gravity and enjoy the fruits of prosperity without having to work an hour a week. This could be great, or terrible. However, I think it is a possibility, and not one to ignore, as radical as it may be. If a mission to Mars is done within the next 20 years, I believe that Mars will be similar to the United States, albeit with less bureaucracy and a more efficient system due to the significantly smaller populace. This is because the first human mission to Mars will likely be done by the United States, so it would make sense that the system on Mars reflects the system that got humans there. This may result in a portion of Mars becoming a territory of the United States, such as Puerto Rico. Maybe China also sends some people to Mars and colonies form, some democratic, others communist. What does all this talk have to do with Earth? Well, it will surely draw attention to the most basic assumption we have about the societies in which we live. This attention may lead many to disagree with some aspects of our societies which could lead to reform. Obviously, we want the best society possible, so reforming our society to better suit the people may become one of the key contributions that Mars has on Earth.

\end{document}